\documentclass[12pt,a4paper]{report}
\usepackage[utf8]{inputenc}
\usepackage[french]{babel}
\usepackage[T1]{fontenc}
\usepackage{graphicx}
\usepackage{geometry}
\usepackage{hyperref}
\usepackage{xcolor}
\usepackage{listings}
\usepackage{tcolorbox}
\usepackage{fancyhdr}
\usepackage{titlesec}
\usepackage{enumitem}
\usepackage{booktabs}
\usepackage{longtable}
\usepackage{array}
\usepackage{tikz}
\usepackage{float}

% Configuration de la page
\geometry{
    top=2.5cm,
    bottom=2.5cm,
    left=3cm,
    right=2.5cm
}

% Configuration des liens
\hypersetup{
    colorlinks=true,
    linkcolor=blue,
    filecolor=magenta,      
    urlcolor=cyan,
    pdftitle={VMManager - CreateVM \& DetailsVM},
    pdfpagemode=FullScreen,
}

% Configuration des listings de code
\lstset{
    basicstyle=\ttfamily\small,
    keywordstyle=\color{blue}\bfseries,
    commentstyle=\color{gray}\itshape,
    stringstyle=\color{red},
    numbers=left,
    numberstyle=\tiny\color{gray},
    stepnumber=1,
    numbersep=10pt,
    backgroundcolor=\color{white},
    showspaces=false,
    showstringspaces=false,
    showtabs=false,
    frame=single,
    rulecolor=\color{black},
    tabsize=2,
    captionpos=b,
    breaklines=true,
    breakatwhitespace=false,
    title=\lstname,
    escapeinside={\%*}{*)},
    morekeywords={*,...}
}

% En-têtes et pieds de page
\pagestyle{fancy}
\fancyhf{}
\fancyhead[L]{\leftmark}
\fancyhead[R]{VMManager - CreateVM \& DetailsVM}
\fancyfoot[C]{\thepage}

% Couleurs personnalisées
\definecolor{primaryblue}{RGB}{33,150,243}
\definecolor{darkblue}{RGB}{13,71,161}
\definecolor{lightgray}{RGB}{245,245,245}

% Style des titres
\titleformat{\chapter}[display]
{\normalfont\huge\bfseries\color{darkblue}}
{\chaptertitlename\ \thechapter}{20pt}{\Huge}

\titleformat{\section}
{\normalfont\Large\bfseries\color{primaryblue}}
{\thesection}{1em}{}

% Boîtes colorées
\newtcolorbox{infobox}{
    colback=blue!5!white,
    colframe=blue!75!black,
    title=Information
}

\newtcolorbox{warningbox}{
    colback=orange!5!white,
    colframe=orange!75!black,
    title=Attention
}

\newtcolorbox{successbox}{
    colback=green!5!white,
    colframe=green!75!black,
    title=Succès
}

% Début du document
\begin{document}

% ============================================================
% PAGE DE GARDE
% ============================================================
\begin{titlepage}
    \centering
    \vspace*{2cm}
    
    {\Huge\bfseries VMManager - CreateVM\par}
    \vspace{0.5cm}
    {\Large Fonctionnalité de Création de Machines Virtuelles\par}
    
    \vspace{2cm}
    
    {\LARGE\bfseries Documentation Technique\par}
    {\Large Interface de Création et Détails de VMs Proxmox\par}
    
    \vspace{3cm}
    
    \begin{tabular}{rl}
        \textbf{Formation :} & EPSI - Administration Réseaux \\
        \textbf{Année :} & 2025-2026 \\
        \textbf{Fonctionnalité :} & CreateVM \\
        \textbf{Technologies :} & Symfony 7, PHP 8.2, JavaScript, Proxmox VE \\
    \end{tabular}
    
    \vfill
    
    {\large \today\par}
\end{titlepage}

% ============================================================
% TABLE DES MATIÈRES
% ============================================================
\tableofcontents
\newpage

% ============================================================
% CHAPITRE 1 : PRÉSENTATION DES FONCTIONNALITÉS
% ============================================================
\chapter{Présentation de la Fonctionnalité CreateVM}

\section{Contexte et Objectifs}

\subsection{Contexte}

La fonctionnalité \textbf{CreateVM} constitue le cœur de la plateforme VMManager, permettant la création de nouvelles machines virtuelles sur une infrastructure Proxmox VE.

Cette fonctionnalité répond au besoin critique de provisionnement rapide et automatisé de ressources virtuelles dans un environnement d'entreprise.

\subsection{Objectifs de la Fonctionnalité}

Les objectifs principaux sont :

\begin{itemize}[leftmargin=2cm]
    \item \textbf{Interface intuitive} : Formulaire moderne pour la création de VMs
    \item \textbf{Configuration flexible} : Ressources personnalisables (CPU, RAM, disque)
    \item \textbf{Validation robuste} : Contrôles côté client et serveur
    \item \textbf{Intégration Proxmox} : Communication transparente avec l'infrastructure
    \item \textbf{Expérience utilisateur} : Processus de création fluide et rapide
\end{itemize}

\subsection{Architecture des Modes}

L'application fonctionne selon deux modes distincts :

\begin{center}
\begin{tikzpicture}[node distance=2cm]
    % Nodes
    \node[draw, rectangle, fill=blue!20, minimum width=4cm, minimum height=1cm] (demo) {Mode Démonstration};
    \node[draw, rectangle, fill=green!20, minimum width=4cm, minimum height=1cm, right of=demo, xshift=3cm] (dynamic) {Mode Dynamique};
    
    % Sub-nodes for demo
    \node[draw, rectangle, fill=blue!10, minimum width=3cm, minimum height=0.8cm, below of=demo, yshift=-0.5cm] (demo-data) {Données statiques};
    \node[draw, rectangle, fill=blue!10, minimum width=3cm, minimum height=0.8cm, below of=demo-data] (demo-ui) {Interface complète};
    
    % Sub-nodes for dynamic
    \node[draw, rectangle, fill=green!10, minimum width=3cm, minimum height=0.8cm, below of=dynamic, yshift=-0.5cm] (dynamic-api) {API Backend};
    \node[draw, rectangle, fill=green!10, minimum width=3cm, minimum height=0.8cm, below of=dynamic-api] (dynamic-real) {Données temps réel};
    
    % Arrows
    \draw[->, thick] (demo) -- node[above] {Migration} (dynamic);
\end{tikzpicture}
\end{center}

% ============================================================
% CHAPITRE 2 : MODE DÉMONSTRATION
% ============================================================
\chapter{Mode Démonstration}

\section{Présentation du Mode Démonstration}

\subsection{Caractéristiques}

Le mode démonstration utilise des \textbf{données statiques} intégrées directement dans le contrôleur PHP pour permettre une démonstration immédiate des fonctionnalités sans dépendance d'un serveur backend.

\begin{infobox}
\textbf{Avantages du mode démonstration :}
\begin{itemize}
    \item Démonstration immédiate des fonctionnalités
    \item Aucune configuration serveur requise
    \item Interface complète et fonctionnelle
    \item Parfait pour les présentations
\end{itemize}
\end{infobox}

\subsection{Comportement sans Serveur Backend}

En mode dynamique sans serveur backend, l'application affiche un \textbf{loader qui tourne indéfiniment}, car :

\begin{itemize}
    \item Les appels API échouent (serveur non disponible)
    \item Le JavaScript tente de se connecter en boucle
    \item L'interface reste en état de chargement
    \item Aucune donnée n'est affichée
\end{itemize}

\begin{warningbox}
\textbf{Problème du mode dynamique sans backend :}
Sans serveur API, l'application reste bloquée sur l'écran de chargement avec un spinner qui tourne sans fin, rendant l'interface inutilisable.
\end{warningbox}

\section{Implémentation du Mode Démonstration}

\subsection{Contrôleur avec Données Statiques}

\begin{lstlisting}[language=PHP, caption=Contrôleur en mode démonstration]
#[Route('/vm/{vmid}', name: 'app_vm_details')]
public function details(int $vmid): Response
{
    // Données VM en dur pour la démonstration
    $vmData = [
        'vmid' => $vmid,
        'name' => 'web-01',
        'status' => 'running',
        'node' => 'proxmox-01',
        'cpus' => 2,
        'maxmem' => 4294967296, // 4GB en bytes
        'mem' => 2147483648, // 2GB utilisés
        'maxdisk' => 53687091200, // 50GB en bytes
        'disk' => 10737418240, // 10GB utilisés
        'uptime' => 86400, // 1 jour en secondes
        'cpu' => 0.25, // 25% CPU
        'netin' => 104857600, // 100MB
        'netout' => 52428800, // 50MB
    ];

    // Données des snapshots en dur
    $snapshots = [
        [
            'name' => 'snapshot-2025-10-15-01',
            'snaptime' => 1728990600,
            'description' => 'Avant mise à jour système',
            'size' => 2684354560
        ],
        // ... autres snapshots
    ];

    // Données des événements en dur
    $events = [
        [
            'date' => '15/10/2025 14:23',
            'type' => 'info',
            'message' => 'VM démarrée',
            'user' => 'admin'
        ],
        // ... autres événements
    ];

    return $this->render('vm/details.html.twig', [
        'vm' => $vmData,
        'snapshots' => $snapshots,
        'events' => $events,
    ]);
}
\end{lstlisting}

\subsection{Template Simplifié}

\begin{lstlisting}[language=HTML, caption=Template en mode démonstration]
<!-- Affichage direct des données statiques -->
<div class="vm-info-card">
    <h1>{{ vm.name }}</h1>
    <span class="vm-status-badge {{ vm.status }}">
        ● En cours● Arrêtée
    </span>
    <div class="vm-specs">
        <div class="vm-spec-item">
            <div class="vm-spec-label">Node</div>
            <div class="vm-spec-value">{{ vm.node }}</div>
        </div>
        <!-- ... autres informations -->
    </div>
</div>

<!-- Données des snapshots -->

<tr>
    <td><strong>{{ snapshot.name }}</strong></td>
    <td>{{ snapshot.snaptime|date('d/m/Y H:i') }}</td>
    <td>{{ snapshot.description }}</td>
</tr>

\end{lstlisting}

% ============================================================
% CHAPITRE 3 : FONCTIONNALITÉ CREATEVM
% ============================================================
\chapter{Fonctionnalité CreateVM}

\section{Interface de Création}

\subsection{Formulaire en 3 Sections}

La page de création propose un formulaire structuré :

\subsubsection{Section 1 : Informations Générales}

\begin{itemize}
    \item \textbf{Nom de la VM} : Champ texte obligatoire
    \item \textbf{Système d'exploitation} : Sélection visuelle parmi 6 options :
    \begin{itemize}
        \item Ubuntu 22.04 (🐧)
        \item Debian 12 (🌀)
        \item CentOS 9 (🔴)
        \item Windows Server (🪟)
        \item Alpine Linux (🏔️)
        \item Custom ISO (⚙️)
    \end{itemize}
\end{itemize}

\subsubsection{Section 2 : Configuration des Ressources}

Configuration via des cartes interactives avec sliders :

\begin{table}[H]
\centering
\begin{tabular}{@{}llll@{}}
\toprule
\textbf{Ressource} & \textbf{Minimum} & \textbf{Maximum} & \textbf{Défaut} \\ \midrule
CPU (vCPU) & 1 & 16 & 2 \\
RAM (GB) & 1 & 64 & 4 \\
Disque (GB) & 10 & 500 & 32 \\ \bottomrule
\end{tabular}
\caption{Configuration des ressources CreateVM}
\end{table}

\subsubsection{Section 3 : Configuration Réseau}

\begin{itemize}
    \item \textbf{Bridge réseau} : vmbr0, vmbr1, vmbr2, vmbr3
    \item \textbf{Adresse IP} : DHCP automatique ou IP statique
\end{itemize}

\subsection{Récapitulatif Dynamique}

Affichage en temps réel de la configuration sélectionnée :

\begin{itemize}
    \item OS sélectionné avec icône
    \item Configuration CPU/RAM/Disque
    \item Paramètres réseau
\end{itemize}

% ============================================================
% CHAPITRE 4 : IMPLÉMENTATION TECHNIQUE
% ============================================================
\chapter{Implémentation Technique}

\section{Architecture Frontend}

\subsection{Structure des Fichiers}

\begin{lstlisting}[language=bash, caption=Structure des fichiers CreateVM]
templates/vm/
└── create.html.twig          # Template principal
src/Controller/
└── ListVMsController.php     # Route /vm/create
assets/
├── app.js                     # Logique JavaScript
└── styles/
    └── app.css               # Styles personnalisés
\end{lstlisting}

\subsection{Contrôleur Symfony}

\begin{lstlisting}[language=PHP, caption=Route de création de VM]
#[Route('/vm/create', name: 'app_vm_create')]
public function create(): Response
{
    return $this->render('vm/create.html.twig');
}
\end{lstlisting}

\section{Validation JavaScript}

\subsection{Validation des Champs}

\begin{lstlisting}[language=JavaScript, caption=Validation du formulaire]
function validateForm() {
    const name = document.getElementById('vm-name').value;
    const os = document.getElementById('vm-os').value;
    const cpu = parseInt(document.getElementById('vm-cpu').value);
    const ram = parseInt(document.getElementById('vm-ram').value);
    const disk = parseInt(document.getElementById('vm-disk').value);
    const bridge = document.getElementById('vm-bridge').value;
    
    // Validation du nom
    if (!name || name.length < 3) {
        showError('Le nom de la VM doit contenir au moins 3 caractères');
        return false;
    }
    
    // Validation des ressources
    if (cpu < 1 || cpu > 16) {
        showError('Le nombre de CPU doit être entre 1 et 16');
        return false;
    }
    
    if (ram < 1 || ram > 64) {
        showError('La RAM doit être entre 1 et 64 GB');
        return false;
    }
    
    if (disk < 10 || disk > 500) {
        showError('L\'espace disque doit être entre 10 et 500 GB');
        return false;
    }
    
    return true;
}
\end{lstlisting}

% ============================================================
% CHAPITRE 5 : MIGRATION VERS LE MODE DYNAMIQUE
% ============================================================
\chapter{Migration vers le Mode Dynamique}

\section{Préparation du Mode Dynamique}

\subsection{Architecture Prête}

L'application est conçue pour supporter facilement le passage au mode dynamique :

\begin{itemize}
    \item \textbf{Service API JavaScript} : Déjà implémenté
    \item \textbf{Endpoints backend} : Structure préparée
    \item \textbf{Gestion JWT} : Authentification prête
    \item \textbf{Flux d'événements} : SSE configuré
\end{itemize}

\subsection{Composants à Activer}

\begin{enumerate}
    \item \textbf{Service API} : Décommenter le code JavaScript
    \item \textbf{Backend API} : Démarrer le serveur backend
    \item \textbf{Configuration} : Variables d'environnement
    \item \textbf{Authentification} : Gestion des tokens JWT
\end{enumerate}

\section{Étapes de Migration}

\subsection{1. Configuration Backend}

\begin{lstlisting}[language=bash, caption=Configuration environnement]
# Variables d'environnement
API_BASE_URL=http://localhost:8001
JWT_SECRET=your-secret-key
PROXMOX_HOST=192.168.1.100
PROXMOX_USER=root@pam
PROXMOX_PASSWORD=your-password
\end{lstlisting}

\subsection{2. Modification du Contrôleur}

\begin{lstlisting}[language=PHP, caption=Contrôleur en mode dynamique]
#[Route('/vm/{vmid}', name: 'app_vm_details')]
public function details(int $vmid): Response
{
    // Mode dynamique : données via API
    return $this->render('vm/details.html.twig', [
        'vmid' => $vmid,
        'api_base_url' => $_ENV['API_BASE_URL'] ?? 'http://localhost:8001',
    ]);
}
\end{lstlisting}

\subsection{3. Activation du JavaScript}

\begin{lstlisting}[language=HTML, caption=Template en mode dynamique]
<!-- Décommenter les sections JavaScript -->
<script src="https://cdn.jsdelivr.net/npm/chart.js"></script>
<script>
const VMID = {{ vmid }};
const API_BASE_URL = '{{ api_base_url }}';
const api = new ApiService(API_BASE_URL);

// Chargement des données via API
async function loadVMData() {
    try {
        const vmResponse = await api.getVMs();
        const vm = vmResponse.data.find(v => v.vmid === VMID);
        
        if (vm) {
            updateVMInfo(vm);
            await loadSnapshots();
            initializeCharts(vm);
            connectEventStream();
        }
    } catch (error) {
        console.error('Erreur chargement données:', error);
    }
}

// Initialisation au chargement de la page
document.addEventListener('DOMContentLoaded', loadVMData);
</script>
\end{lstlisting}

\subsection{4. Service API Backend}

\begin{lstlisting}[language=PHP, caption=Endpoints API backend]
#[Route('/api/v1/vms', methods: ['GET'])]
public function getVMs(): JsonResponse
{
    $vms = $this->proxmoxService->getAllVMs();
    return $this->json(['success' => true, 'data' => $vms]);
}

#[Route('/api/v1/vms/{vmid}', methods: ['GET'])]
public function getVM(int $vmid): JsonResponse
{
    $vm = $this->proxmoxService->getVM($vmid);
    return $this->json(['success' => true, 'data' => $vm]);
}

#[Route('/api/v1/vms/{vmid}/start', methods: ['POST'])]
public function startVM(int $vmid): JsonResponse
{
    $result = $this->proxmoxService->startVM($vmid);
    return $this->json(['success' => $result]);
}
\end{lstlisting}

\section{Avantages du Mode Dynamique}

\subsection{Fonctionnalités Activées}

\begin{itemize}
    \item \textbf{Données temps réel} : Métriques actualisées automatiquement
    \item \textbf{Actions interactives} : Démarrage/arrêt des VMs
    \item \textbf{Graphiques dynamiques} : Évolution des performances
    \item \textbf{Événements en direct} : Flux SSE des opérations
    \item \textbf{Gestion des snapshots} : Création et restauration
\end{itemize}

\subsection{Performance et Monitoring}

\begin{itemize}
    \item \textbf{Temps de réponse} : < 2 secondes
    \item \textbf{Mise à jour automatique} : Toutes les 5 secondes
    \item \textbf{Gestion d'erreurs} : Retry automatique
    \item \textbf{Authentification} : JWT avec renouvellement
\end{itemize}

% ============================================================
% CHAPITRE 6 : COMPARAISON DES MODES
% ============================================================
\chapter{Comparaison des Modes}

\section{Tableau Comparatif}

\begin{table}[H]
\centering
\begin{tabular}{@{}p{3cm}p{6cm}p{6cm}@{}}
\toprule
\textbf{Aspect} & \textbf{Mode Démonstration} & \textbf{Mode Dynamique} \\ \midrule
Données & Statiques (PHP) & Temps réel (API) \\
Configuration & Aucune & Backend requis \\
Performance & Instantanée & Dépendante réseau \\
Fonctionnalités & Affichage uniquement & Actions complètes \\
Démonstration & Parfaite & Complète \\
Production & Non adapté & Recommandé \\
Maintenance & Minimale & Standard \\
Évolutivité & Limitée & Totale \\ \bottomrule
\end{tabular}
\caption{Comparaison des modes de fonctionnement}
\end{table}

\section{Recommandations}

\subsection{Mode Démonstration}

\textbf{Utilisation recommandée :}
\begin{itemize}
    \item Présentations et démonstrations
    \item Développement d'interface
    \item Tests de design
    \item Formation utilisateurs
\end{itemize}

\subsection{Mode Dynamique}

\textbf{Utilisation recommandée :}
\begin{itemize}
    \item Environnement de production
    \item Gestion réelle des VMs
    \item Monitoring en continu
    \item Intégration avec infrastructure
\end{itemize}

% ============================================================
% CHAPITRE 7 : GUIDE DE MIGRATION
% ============================================================
\chapter{Guide de Migration}

\section{Checklist de Migration}

\subsection{Prérequis}

\begin{enumerate}
    \item \textbf{Serveur backend} : API Symfony configurée
    \item \textbf{Proxmox VE} : Infrastructure accessible
    \item \textbf{Authentification} : JWT configuré
    \item \textbf{Réseau} : Communication frontend/backend
\end{enumerate}

\subsection{Étapes de Migration}

\begin{enumerate}
    \item \textbf{Configuration environnement} : Variables d'API
    \item \textbf{Modification contrôleur} : Suppression données statiques
    \item \textbf{Activation JavaScript} : Décommenter le code
    \item \textbf{Test des fonctionnalités} : Vérification complète
    \item \textbf{Déploiement} : Mise en production
\end{enumerate}

\section{Script de Migration}

\begin{lstlisting}[language=bash, caption=Script de migration automatique]
#!/bin/bash

echo "=== Migration vers le mode dynamique ==="

# 1. Configuration environnement
echo "Configuration des variables d'environnement..."
cp .env.example .env
echo "API_BASE_URL=http://localhost:8001" >> .env

# 2. Modification du contrôleur
echo "Modification du contrôleur..."
sed -i 's/\/\/ Mode démonstration/Mode dynamique/g' src/Controller/ListVMsController.php

# 3. Activation du JavaScript
echo "Activation du JavaScript..."
sed -i 's/<!-- <script>/<script>/g' templates/vm/details.html.twig
sed -i 's/<\/script> -->/<\/script>/g' templates/vm/details.html.twig

# 4. Test des fonctionnalités
echo "Test des fonctionnalités..."
php bin/console cache:clear

echo "=== Migration terminée ==="
\end{lstlisting}

% ============================================================
% CONCLUSION
% ============================================================
\chapter{Conclusion}

\section{Bilan des Fonctionnalités}

Les fonctionnalités \textbf{CreateVM} et \textbf{DetailsVM} offrent :

\begin{itemize}
    \item \textbf{Mode démonstration} : Interface complète avec données statiques
    \item \textbf{Mode dynamique} : Fonctionnalités temps réel avec API backend
    \item \textbf{Migration facile} : Passage simple entre les modes
    \item \textbf{Architecture flexible} : Adaptation aux besoins
\end{itemize}

\section{Perspectives d'Évolution}

\begin{itemize}
    \item \textbf{Mode hybride} : Combinaison des deux approches
    \item \textbf{Cache intelligent} : Optimisation des performances
    \item \textbf{Offline mode} : Fonctionnement sans connexion
    \item \textbf{API avancée} : Endpoints supplémentaires
\end{itemize}

\begin{successbox}
Les fonctionnalités CreateVM et DetailsVM constituent une base solide pour la gestion de machines virtuelles, avec une architecture flexible permettant l'adaptation aux différents contextes d'utilisation.
\end{successbox}

\end{document}